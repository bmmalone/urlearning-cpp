\documentclass[a4paper,10pt]{article}

% colored text
\usepackage{color}

% include eps, pdf graphics
\usepackage{graphicx}

% use "H" for floats
\usepackage{float}

% avoid "too many unprocessed floats" error
\usepackage{morefloats}

% FloatBarrier to ensure figures do not jump to different sections
\usepackage{placeins}

% properly handle spaces after defines
\usepackage{xspace}

% in case we need to span rows in our tables
\usepackage{multirow}

% tables across multiple pages
\usepackage{ltablex}

% nice-looking tables
\usepackage{booktabs}

% easy centering for tables
\newcolumntype{Y}{>{\centering\arraybackslash}X}

% math
\usepackage{algorithm}
\usepackage{algorithmicx}
\usepackage[noend]{algpseudocode}
\usepackage{amsmath,amsthm,amssymb}

% more math
\newcommand*{\defeq}{\mathrel{\vcenter{\baselineskip0.5ex \lineskiplimit0pt
                     \hbox{\scriptsize.}\hbox{\scriptsize.}}}%
                     =}

\DeclareMathOperator*{\argmax}{arg\,max}
\DeclareMathOperator*{\argmin}{arg\,min}

\newcommand{\BigO}[1]{\ensuremath{\operatorname{O}\left(#1\right)}}
\newtheorem{definition}{Definition}
\newtheorem{theorem}{Theorem}

% mono (\|), bold (\!) and fancy (\*) letters
\def\|#1{\ensuremath{\mathtt{#1}}}
\def\!#1{\ensuremath{\mathbf{#1}}}
\def\*#1{\ensuremath{\mathcal{#1}}}

\newcommand\todo[1]{\textcolor{red}{[TODO: #1]}}

% custom commands
\newcommand\uprime{\ensuremath{\!U^{\prime}}\xspace}
\newcommand\Astar{A$^{*}$\xspace}

\title{Pseudocode for URLearning algorithms}
\author{Brandon Malone}

\begin{document}


\maketitle

\begin{abstract}
This document describes the algorithms implemented in the URLearning software
package. The document \emph{does not} necessarily describe the inner behavior
of the solvers, which often contain optimizations to improve performance.
\end{abstract}

\section{\Astar Search Algorithm}

The \Astar search algorithm uses a best-first expansion policy to explore the
order graph. Algorithm~\ref{alg:astar} gives a high-level overview of this
search strategy. The primary operations are (immediate) duplicate detection
and maintaining the priority queue.

In practice, the \|{contains} operations are implemented using a single hash
table which contains all generated nodes. Flags indicate whether the nodes
are in the $open$ or $closed$ lists.

Further, (pointers to) nodes in the $open$ list are also maintained in a
priority queue, which is implemented on top of a standard heap. The \|{update}
operation leverages the observation that node priorities only ever improve; if
a duplicate is worse, it is simply ignored.

\begin{algorithm}[]
\caption{\Astar Search Algorithm}
\begin{algorithmic}[0]
%\Require full or 
%\Ensure an optimal Bayesian network $G$

\Function {astar}{{\small sparse parent graphs with $BestScore(\cdot)$, 
        admissible heuristic $h$}}

    \State $start \leftarrow \emptyset$
    \State $Score(start) \leftarrow 0$
    \State push($open, start, h(start)$

    \While {len($open$) $> 0$}
        \State $\!U \leftarrow $pop($open$)
	\If {\!U is goal} \Comment{A shortest path is found}
            \State $\*N^{*} \leftarrow$ construct a network from the shortest path
	    \State \Return $\*N^{*}$
        \EndIf

        \State put($closed, \!U$)

	\For {each $X \in \!V \setminus \!U$} \Comment{Generate successors}
            \State $\uprime \leftarrow {\bf U} \cup \{X\}$
	    \State $g \leftarrow BestScore(X, {\bf U}) + Score({\bf U})$
%			
	    \If {contains($closed, \uprime $)} \Comment{Closed list DD}
                \If{$g< Score(\uprime)$} \Comment{reopen node}
                    \State remove($closed, \uprime$)
                    \State push ($open, \uprime, g + h(\uprime)$)
                    \State $Score(\uprime) \leftarrow g$
                \EndIf
            \ElsIf {contains($open, \uprime $)} \Comment{Open list DD}
                \If{$g< Score(\uprime)$} \Comment{better path}
                    \State update($open, \uprime, g + h(\uprime)$)
                    \State $Score(\uprime) \leftarrow g$
                \EndIf
            \Else \Comment{New node}
                \State push ($open, \uprime, g + h(\uprime)$)
                \State $Score(\uprime) \leftarrow g$
            \EndIf
        \EndFor

    \EndWhile
\EndFunction

\end{algorithmic}
\label{alg:astar}
\end{algorithm}


\end{document}
